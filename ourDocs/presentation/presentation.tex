\documentclass{beamer}
\usetheme{metropolis}
\title{LEGO Mindstorms Line Following Project}
\author{Aniruddha Pal, Markus Wiktorin}
\date{23rd of September 2016}
\institute{University of Applied Science, Bonn-Rhein-Sieg}
\begin{document}
	\maketitle
	\begin{frame}
		\frametitle{Task}
		Program a LEGO robot which can follow a line.
		\\
		Light conditions and track can change, the robot should work under different conditions.
	\end{frame}
	\begin{frame}
		\frametitle{Approaches}
		\begin{enumerate}
			\item One light sensor in front of the robot
			\item Two light sensors in front of the robot
			\item Two light sensors in front of the robot using PID
			\item Two light sensors, ultrasonic sensor and touch sensor in front of the robot using PID
		\end{enumerate}
	\end{frame}
	\begin{frame}
		\frametitle{1. Approach (one light sensor)}
		
	\end{frame}
	\begin{frame}
		\frametitle{2. Approach (two light sensors)}
		
	\end{frame}
	\begin{frame}
		\frametitle{3. Approach (two light sensors + PID)}
		
	\end{frame}
	\begin{frame}
		\frametitle{4. Approach (two light, ultrasonic, touch sensor + PID)}
		
	\end{frame}
\end{document}